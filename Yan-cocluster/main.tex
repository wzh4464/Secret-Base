%%%
 % File: /main.tex
 % Created Date: Sunday September 10th 2023
 % Author: Zihan
 % -----
 % Last Modified: Sunday, 10th September 2023 6:51:15 pm
 % Modified By: the developer formerly known as Zihan at <wzh4464@gmail.com>
 % -----
 % HISTORY:
 % Date      		By   	Comments
 % ----------		------	---------------------------------------------------------
%%%

\documentclass[12pt]{article}
\usepackage[a4paper, margin=0.8in]{geometry} % Adjust margins
\usepackage{setspace} % Adjust line spacing
% set spacing to 1.0
\setstretch{1.0} 
\usepackage{amsmath}
\usepackage{amssymb}
% \usepackage{mathptmx} 
% theorems
\usepackage{amsthm}


\title{Probability of detecting co-clusters and setting parameters}
\author{}
\date{}
\begin{document}
\maketitle
\section{Notation}
\begin{itemize}
    \item $A \in \mathbb{R}^{M \times N}$ is a matrix with $K$ co-clusters (co-cluster set $C = \{C_k\}_{k=1}^K$);
    \item $A$ is partitioned into $m \times n$ blocks, each block has size $P_i \times Q_j$, that is, $M=\sum_{i=1}^m P_i$ and $N=\sum_{j=1}^n Q_j$;
    \item thus block set $B = \{B_{(i,j)}\}_{i=1,j=1}^{Q_m,Q_n}$;
          % \item $(i,j) \in \mathbb{N}/Q_m \times \mathbb{N}/Q_n$ is the index of a block;
          % sub-co-cluster $k$ 的落到block $(i,j)$ 的size 是 $M_i^{(k)} \times N_i^{(k)}$;
    \item the size of sub-co-cluster $C_k \in \mathbb{R}^{M^{(k)} \times N^{(k)}}$ that falls into block $B_{(i,j)}$ is $M_{(i,j)}^{(k)} \times N_{(i,j)}^{(k)}$;
    \item $T_m$ is the minimum number of rows, $T_n$ is the minimum number of columns.
\end{itemize}

\section{Probability}
% P(Mi<Tm, Ni<Tn for all i in partition j | Qm, Qn, Ms, Ns)
Consider co-cluster $C_k$,
% P(M_{(i,j)}^{(k)} = i)
\begin{align*}
    P(M_{(i,j)}^{(k)} = \alpha) & = \frac{\binom{M_k}{\alpha} \binom{M-M_k}{P_i-\alpha}}{\binom{M}{P_i}} \\
    % p[x_] := Binomial[Mk, a] * Binomial[M-Mk, P-a] / Binomial[M, P]
    P(N_{(i,j)}^{(k)} = \beta)  & = \frac{\binom{N_k}{\beta} \binom{N-N_k}{Q_j-\beta}}{\binom{N}{Q_j}}
    % q[y_] := Binomial[Nk, b] * Binomial[N-Nk, Q-b] / Binomial[N, Q]
\end{align*}
The tail probability of $M_{(i,j)}^{(k)}$ and $N_{(i,j)}^{(k)}$ are
\begin{align*}
    P(M_{(i,j)}^{(k)} < T_m) & = \sum_{\alpha=1}^{T_m-1} P(M_{(i,j)}^{(k)} = \alpha) \\
                             & \le \exp(-2 (s_i^{(k)})^2 P_i)
\end{align*}
where $s_i^{(k)} = \cfrac{M_k}{M}-\cfrac{T_m-1}{P_i}$, and
\begin{align*}
    % &\le \sum_{\alpha=1}^{T_m-1} \frac{\binom{M_k}{\alpha} \binom{M-M_k}{P_i-\alpha}}{\binom{M}{P_i}} \\
    P(N_{(i,j)}^{(k)} < T_n) & = \sum_{\beta=1}^{T_n-1} P(N_{(i,j)}^{(k)} = \beta) \\
                             & \le \exp (-2 (t_j^{(k)})^2 Q_j)
\end{align*}
where $t_j^{(k)} = \cfrac{N_k}{N}-\cfrac{T_n-1}{Q_j}$.

The joint probability of $M_{(i,j)}^{(k)}$ and $N_{(i,j)}^{(k)}$ are
\begin{align*}
    P(M_{(i,j)}^{(k)} < T_m, N_{(i,j)}^{(k)} < T_n) & = \sum_{\alpha=1}^{T_m-1} \sum_{\beta=1}^{T_n-1} P(M_{(i,j)}^{(k)} = \alpha) P(N_{(i,j)}^{(k)} = \beta) \\
    % pq[x_, y_] := Sum[p[a] * q[b], {a, 1, x-1}, {b, 1, y-1}]
                                                    & \le \exp[-2 (s_i^{(k)})^2 P_i + -2 (t_j^{(k)})^2 Q_j]
\end{align*}
If $P_i = p$ and $Q_j = q$ for all $i$ and $j$, then

Suppose event $\omega_k$ is that co-cluster $C_k$ can't be find in any block $B_{(i,j)}$, then
\begin{align*}
    P(\omega_k) & = \prod_{i=1}^m \prod_{j=1}^n P(M_{(i,j)}^{(k)} < T_m, N_{(i,j)}^{(k)} < T_n)                    \\
                & \le \prod_{i=1}^m \prod_{j=1}^n \exp\{-2 \left[ (s_i^{(k)})^2 P_i + (t_j^{(k)})^2 Q_j \right] \} \\
                & = \exp\{-2 \sum_{i=1}^m \sum_{j=1}^n \left[ (s_i^{(k)})^2 P_i + (t_j^{(k)})^2 Q_j \right] \}     \\
\end{align*}

If $P_i = p$ and $Q_j = q$ for all $i$ and $j$, then
\begin{align*}
    s_i^{(k)} & = s^{(k)} = \frac{M_k}{M}-\frac{T_m-1}{p} \\
    t_j^{(k)} & = t^{(k)} = \frac{N_k}{N}-\frac{T_n-1}{q}
\end{align*}

\begin{align*}
    P(\omega_k) & \le \exp \left\{ -2 [pm (s^{(k)})^2 + qn (t^{(k)})^2] \right\} \\
\end{align*}


And if we do $T_p$ times of random sampling, the Probability of detecting the co-cluster is
\begin{align*}
    P & = 1 - P(\omega_k)^{T_p}                                              \\
      & \ge 1 - \exp \left\{ -2 T_p [pm (s^{(k)})^2 + qn (t^{(k)})^2] \right\} \\
\end{align*}
according to which, we can set $m, n, p, q, T_m, T_n$ and $T_p$ to ensure the probability of detecting the co-cluster is larger than a given threshold.

\end{document}