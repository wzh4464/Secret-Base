\documentclass{article}
\usepackage[utf8]{inputenc}
\usepackage{amsmath, amssymb}
% \usepackage[backend=biber,style=verbose-trad2]{biblatex}
% IEEE style
\usepackage[backend=biber,style=ieee]{biblatex}
\bibliography{ref}

\title{Exploring Discrete Approximations of Manifolds in Computational Geometry}
\author{Biber}
\date{}

\begin{document}

\maketitle

\section{Introduction}
In this paper, we delve into the relationship between continuous manifolds and their discrete approximations, particularly focusing on the mesh functor \( g_\theta \) and its role in computational geometry.

\section{Mesh Functor \( g_\theta \) and its Properties}
The mesh functor \( g_\theta \), given a parameter \( \theta \), discretizes a continuous manifold \( M \) into a finite triangular mesh \( P \). This process is crucial in computational applications where exact manifold representations are infeasible. 

\subsection{Definition and Continuity}
The functor \( g_\theta \) can be formally defined as follows:
\[ g_\theta : \mathbf{Man} \rightarrow \mathbf{SimpCplx} \]
where \( \mathbf{Man} \) denotes the category of 3-dimensional manifolds and \( \mathbf{SimpCplx} \) the category of simplicial complexes. This functor preserves essential topological and geometric features of \( M \), albeit in a discretized form.

\subsection{Error Analysis}
An important aspect of \( g_\theta \) is the error introduced during the discretization. The Gromov-Hausdorff distance provides a measure of this error, offering insights into the fidelity of \( P \) as an approximation of \( M \).

\section{Comparative Analysis}
\subsection{Manifold \( M \) versus Mesh \( P \)}
The fidelity of \( P \) in representing \( M \) is crucial for algorithms in computational geometry. Despite the inherent approximation, \( P \) maintains most of the critical geometric properties of \( M \), allowing for effective algorithmic processing.

\subsection{Algorithmic Considerations}
The discrete nature of \( P \) introduces unique computational challenges and opportunities. Algorithms must be adapted to handle the discrete structure efficiently, balancing accuracy and computational complexity.\autocite{coeurjolly2001DiscreteCurvatureBased}

\section{Conclusion}
Our exploration underscores the significance of the mesh functor \( g_\theta \) in bridging the gap between theoretical manifolds and their practical computational counterparts. This understanding is pivotal for advancing methods in computational geometry and related fields.
\autocite{wu2020DeltaGradRapidRetraining}

\printbibliography

\end{document}